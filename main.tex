\documentclass[apj]{emulateapj}
\usepackage{amsmath,amssymb,amstext,color,colortbl}
\usepackage{units}
\usepackage[breaklinks,colorlinks,citecolor=blue,linkcolor=magenta]{hyperref}
\usepackage[all]{hypcap} %Links go to figures;breaks on deluxetables
\usepackage{natbib}
\bibliographystyle{apj}

\shorttitle{Lightcurves of Peculiar Type II-P SNe}
\shortauthors{Lusk \& Baron}

\begin{document}

\title{TEST}
\author{Jeremy A. Lusk\altaffilmark{1} and Eddie Baron} \affil{Homer L. Dodge Department of Physics and Astronomy, University of Oklahoma, Norman, OK 73019}

\altaffiltext{1}{Math/Natural Science Division, Midland College, Midland, TX 79705}

\begin{abstract}
We examine the bolometric lightcurves of six Type~II-P supernovae which are thought to originate from blue supergiant progenitors. Thank you. How are you?
Fine, thanks.

\end{abstract}

\section{Introduction}

Alvares The bolometric luminosity of a supernova is the total radiant luminosity, typically measured in erg s$^{-1}$.
The variation of this luminosity with time after explosion is the lightcurve, and is an important quantity in the study of transient objects such as supernovae.
Supernovae are categorized by the shapes of their lightcurves. %TODO: cite Filippenko's review paper
The shapes of lightcurves also reveal important information about a supernova.
Hydrodynamic models of expanding supernova ejecta output bolometric lightcurves which can be compared with those of observed supernovae.
From these comparisons, the model is used to estimate the mass and structure of the progenitor, the total energy of the explosion, and the amount of radioactive $^{56}$Ni synthesized in the supernova.

Determining the properties of a supernova progenitor by matching its observed bolometric lightcurve to one calculated by a hydrodynamic model is prediacted on the ability to determine the bolometric lightcurve of a supernova in the first place.
%TODO: Talk about the challenges of obtaining a bolometric lighcurve from observations. Motivate the need for comparing the accuracy of different methods.

\section{Methods}

The methods for calculating a bolometric lightcurve from observed photometric magnitudes can be classified into two categories: direct integration and polynomial fitting.

Direct integtration uses only the observed photometry, converting broad-band filter magnitudes to monochromatic fluxes at wavelengths representative of the filters.
These fluxes are then integrated, typically using the trapezoidal method, to generate a value often referred to as the quasi-bolometric flux.
This quasi-bolmetric flux represents only the observed portion of the total spectral energy distribution of the supernova, and does not include flux in the UV or IR.
The quasi-bolometric flux is typically enhanced by UV and IR corrections - estimates of the missing flux blueward and redward of the observed flux. %TODO JL: Organize a list of references for UV and IR corrections
Usually, these corrections are made by fitting a blackbody to the observed flux, and integrating that blackbody function from the reddest observed wavelength to infinity, and from the bluest observed wavelength to zero. % TODO JL: What's the earliest reference for this method of correcting the IR flux?
Because the SED of supernovae are known to depart significantly from that of a blackbody due to line blanketing in the UV, there are a variety of ways that different groups handle UV corrections, which will be discussed in detail below. %TODO JL: Include good reference for UV line blanketing

Polynomial fitting methods use the bolometric lightcurves of well-observed supernovae (usually calculated using the direct integration technique as mentioned above) to find correlations between an observable quantity such as color and the bolometric correction $BC = m_{bol} - (V - A_V)$.
With this, magnitudes in a filter band can be converted to bolometric magnitudes, and then into bolometric luminosities.
By finding polynomails which describe the relationship between color and bolometric correction, the bolometric luminosity of a less well-observed supernova can be calculated simply my making color observations and a distance estimate.
This assumes, of course, that the same relationship found between the color and bolometric correction of the template supernova exists for the less well-observed supernova.
Is is therefore important to use several different well-observed supernovae to establish the polynomials used to transform color into a bolometric correction.

\section{Results}

\subsection{Nickel Mass Estimates}

The post-plateau luminosity of a Type II~P supernova comes primarily from energy deposited by the gamma-rays produced in the decay chain $^{56}$Ni $\rightarrow$ $^{56}$Co $\rightarrow$ $^{56}$Fe.
We use the $\gamma$-ray specific energy for pure $^{56}$Ni from \cite{sutherland_models_1984}:

\begin{equation}
    s = 3.90 \times 10^{10} e^{-\gamma_1 t} + 6.78 \times 10^{9} \left( e^{-\gamma_2 t} - e^{\gamma_1 t} \right)
\end{equation}
where $s$ is in erg s$^{-1}$ g$^{-1}$. The constants $\gamma_1 = \unit [1.32\times10^{-6}]{s^{-1}}$, and $\gamma_2 = \unit[1.02\times10^{-7}]{s^{-1}}$. $\gamma_1$ are the decay rates of $^{56}$Ni and $^{56}$Co, corresponding to half-lives of \unit[6.08]{d} and \unit[78.65]{d}, respectively.

The nickel mass ejected by the supernova can be estimated by fitting the luminosity of the post-plateau tail with the equation

\begin{equation}
    L_{Ni} = sM_{Ni}
\end{equation}
In our calculations, we used the \textsc{curve\_fit} function from \textsc{scipy}\footnote{\url{http://www.scipy.org}} \citep{jones_scipy:_2001}

For SN 1998A, the reported mass of $^{56}$Ni in \cite{pastorello_sn_2005} is \unit[0.11]{M$_{\odot}$}.


\bibliography{bsg_lbol}



\end{document}